\section{Legal and Ethical Issues}\label{sec:legalandethicalissues}


The privacy policy of a work must be at least as protective of the information as the most restrictive of the sources used. As a matter of fact, all the data sources used along the work need to be taken into account in the study of its legal and ethical issues.

The issues related with data protection copyright and intellectual property legislation can be very tedious, specially if the source is a private company. To enhance the understanding, and even though some of the sources do not present copyright-licenses of this type, the icons used and released by the non-profit organisation Creative Commons will be used in this section. The Creative Commons licenses  require attribution for copying, sharing and verbatim uses but there are other restrictions that may apply (see Table \ref{table_5}).

\begin{table}[]
\centering
\caption{ The four types of restrictions that may apply to a Creative Common license.}
\label{table_5}
\begin{tabular}{ p{1cm} p{1cm} p{10cm}}
    \begin{minipage}{.1\textwidth}
      \includegraphics[scale=.07]{images/cc_by.png}
    \end{minipage} & \textbf{BY} & \textbf{Attribution} - credit to the creator is required                                                                                                             
    \\
 \begin{minipage}{.1\textwidth}
      \includegraphics[scale=.07]{images/cc_sa.png}
    \end{minipage} & \textbf{SA} & \textbf{Share Alike} - distribution of the work needs to be under the same licence of the work (e.g. a copy made from a Creative Commons work cannot be copyrighted) 
    \\
 \begin{minipage}{.1\textwidth}
      \includegraphics[scale=.07]{images/cc_nd.png}
    \end{minipage} & \textbf{ND} & \textbf{No Derivatives} - cannot make changes to or remix the work                                                                                                    \\
 \begin{minipage}{.1\textwidth}
      \includegraphics[scale=.07]{images/cc_nc.png}
    \end{minipage} & \textbf{NC} & \textbf{Non-commercial} - commercial use of the work is not permitted                                                                                               
\end{tabular}
\end{table}

\subsection{Meetup API}

The information regarding the use that can be done with any information extracted from the Meetup API is detailed in 5.8 \textit{API License} of the \textit{Terms of Service} section and expanded in the some other sections to which a link is attached. The introduction of the subsection is attached as follows. From this paragraph (and from the rest of the section), it is clear that restrictions \textbf{BY}, \textbf{SA}, \textbf{ND} apply.

\vspace{0.5cm}
"Meetup grants to you a limited, non-exclusive, non-transferable, non-sublicensable, revocable license to use the Meetup application programming interface, including data or other Content made available via the Meetup API, (...) solely to facilitate the development of event and group related applications using Platform data and developer tools."
\vspace{0.5cm}

Notice however that commercial use of the work is not said to be prohibited. Section \textit{Meetup API License Guidelines} details what kind of applications and under what conditions (i.e., "enhance the Meetup experience or create specialized versions") the commercial use of the information provided by this platform is allowed. Not surprisingly, though, the conditions given already restrict a lot the possibility of making any commercial use of the data, so restriction \textbf{NC} could almost be added to the restrictions list too.

\subsection{Google Maps API}

The restrictions in the use of this API are considerably more complicated than the ones in the Meetup case, and are spread out in sections 6 to 12 of the \textit{Google Maps/Google Earth APIs Terms of Service}. The summary of restrictions is found in section 8.2, \textit{Service License}, where it is made clear that the license to use the Service is non-sublicensable. A part from some restrictions, section 9.1 \textit{.1 Free, Public Accessibility to Your Maps API Implementation.} specifies the terms under which a commercial use of the Service would be allowed. Attribution and non derivatives restrictions are also specified in sections 9.4 and 10.5 respectively, so the whole set of restrictions is finally  \textbf{BY}, \textbf{SA}, \textbf{ND} and \textbf{NC} in some cases.

\subsection{Different sources of \textit{.geoJSON} files}



\subsection{Wikipedia}

As explained in the \textit{Terms} section of Wikimedia, the contents of this platform are only restricted by \textbf{BY} and \textbf{SA}.
